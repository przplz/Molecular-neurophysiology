\documentclass[a4paper, 12pt]{book}
\usepackage[english]{babel}
\usepackage[T1]{fontenc}
\usepackage[utf8]{inputenc}
\usepackage{todonotes}
\usepackage{caption}
\usepackage{subfig}
\usepackage{textcomp}
\usepackage{float}
\captionsetup{format=hang,labelfont={sf,bf}}
\usepackage{amsmath, amssymb}
\usepackage{xspace}
\usepackage[hidelinks]{hyperref}
\usepackage{mhchem}


\newcommand{\freccia}{\ensuremath{\rightarrow}}
\newcommand{\lfreccia}{\ensuremath{\longrightarrow}}
\newcommand{\gradi}{\textcelsius\xspace}
\newcommand{\ang}{\AA \xspace}



\begin{document}
\author{Elisa Nerli}
\title{Molecular Neurophysiology}
\maketitle
\newpage
\tableofcontents
\newpage



\chapter{Lesson 1: Cell Communication}

\section{Introduction}
Paola Lorenzon, coordinator of this course, that is divided into three parts:
\begin{itemize}
\item{Signal Transduction - Paola Lorenzon, 24h of lessons}
\item{Synaptic Transmission in SNC - Marina Sciancalepore, 24h of lessons}
\item{Principles of synaptic transmission SNP - Annalisa Bemareggi, 16h of lessons}
\end{itemize}

Frequency is mandatory for 43 hours to be admitted to the final exam.

Exam: Oral + activity before the oral exam (if we are interested), which consist in a written test, we have to read an article and write an abstract. The maximum mark is 1.5, the minimum is 0. When you pass the oral exam, they look at it and sum up the mark of the oral.

Textbook: none, or Kandel.

Slides and material: Moodle, psw= GHK2015


\section{Cell communication}
Cell communication is needed in every multicellular organism, because cells are specialized and organized in different tissues. This does not mean that unicellular organisms don't communicate: they communicate in different ways. The end point of communication in to \emph{maintain the homeostasis}, or the wellness of the organism. The cell communication depends on the distance between the cell elements which are communicating: it can be long or absent,  like in cells of the same tissue, in which the plasmalemma is in contact (this is the simplest way).

A direct contact between 2 cells means that material can pass from one cell to the other one, like ions, small molecules, second messengers. The plasmalemma is a barrier that has to be overcome: there are different ways to make adjacent cells very very close:
\begin{itemize}
\item{GAP junction: made of \emph{connexons} (connessoni), very big ion channels made of 6 units, called \emph{connexines}. Each connexon is located on the membrane of a cell, and in front of the connexon of the adjacent cell. So, ions and small molecules can diffuse thanks to 2 driven forces: difference of concentration, if the molecule is non charged, and  difference of membrane potential if the molecule is an ion (so it is charged). Ions will be move also because of the different distribution in the two cells. We can't identify nor the sender (cell 1) neither the receiver (cell 2) because communication is bidirectional. So, we need chemical gradient and electrical gradient. The endpoint is a \emph{variation of cytoplasm composition} of the receiving cell, to start a biological response.}
\item{Electrical synapses: can occur in small neurons, they are very very fast in communication and in the transmission. If the two communicating elements are excitable, the messengers are ions, so there is chemical or electrical gradient, necessary to start communication. The receiving cell changes membrane potential: it can be depolarized or hyperpolarized. }
\end{itemize}

When there is a distance between the cells, the only way is to send something soluble in the extracellular milieu, an hydrophilic molecule, small to have a high diffusion rate. These molecules are called \emph{first messengers}, or \emph{ligand}. There is a specialization both in the sender and the receiver: the sender has all the machinery for the syntesis and the secretion of the first messenger; the receiver has to have all the machinery needed to detect the messenger in the extracellular milieu. These cells can be identified by experimental evidences.

The receiver must express receptors (proteins) for the soluble molecule.

We can discriminate cell communication in:
\begin{itemize}
\item{If the distance is short: \emph{diffusion}. The secretory cell create a chemical gradient because of secretion. The messenger starts to move far away to the secretory cell and just in front of it It will find the receptor. This is \emph{paracrine communication}: chemical synapses are an examples of this communication, the most common in humans, and the molecules are \emph{neurotransmitter}.}
\item{If the distance is long: diffusion is not so efficient! The communication uses the blood vessels as a sort of very large road in which the first messenger can increase the speed thanks to blood flux rate. This communication is used for \emph{neurohormones}. This communication depends also on the well-being of the blood vessels. In this case, in the receiver cell the biological response consist in detect the molecule, but apparently nothing occur in the cytoplasm. That is not true: the first messenger binding to the receptor start the process of \emph{Signal transduction}}
\end{itemize} 

Signal transduction are all the process that occurs after the binding of the ligand to the receptor: in the cytoplasm of the receiver something will happen, something changes. The external signal is converted into an intracellular signal, which is different, is another molecule or something else, like a variation of the concentration of a second messenger inside (like Ach and cAMP), that will start the biological response.

\subsection{Signal transduction requires receptor}
Receptor is usually a glycoprotein. Each receptor has a different geometric profile and different binding sites, also a different 3D organization. Not just the shape, but also the chemical properties of the binding site can be different, depending on the aa. The answer of the target cell is a specific phenomenon.  Also, if the communication is on long distance, how many cell can the neurohormone touch? A lot!!!!

The \emph{specificity} of the communication depends on the chemical properties and the geometry of the ligand, but also on the chemical properties and the geometry of the receptor, expressed exclusively on that cell!

The \emph{receptor affinity for the ligand} is another aspect: the affinity may be low or high. If it is high, the problem is the reversibility of the communication! The equilibrium means not that the products of the reaction are favored. It is important that all the biological process are \emph{reversible}. The communication must depend on the concentration of ligand present at a certain time: if it goes down, the communication must stops. If is high, it must goes. The receptor affinity is not too high, not to low \lfreccia reversibility of the cell communication. This allow a time-limited activation of the target cell.

The \emph{localization of the receptor} is also important: there are receptor soluble in the cytoplasm or in the nucleus. The membrane is not required for the biological activity of the receptor: there are membrane receptors or intracellular receptor. The identification of this localization is important because if you know it, you can predict the machinery activated by the transduction.  If the receptor is on the surface, the ligand has to be hydrophilic; if the receptor is intracellular, the ligand has to be lipophilic.

The \emph{structure} helps us to predict the machinery and the transduction mechanism. If the ligand rise and bind the receptor, which is a ion channel, there will be a change in the permeability for that ion. We can predict that this is an ion-channel receptor.

There is difference between ligand, antagonist and agonist. Ligand is a first messenger,  a physiological ligand like a neurotransmitter. Agonist is a non-physiological messenger present in nature or synthetic, which activates the receptor and is able to use the biological response of that receptor. It can be a toxin or a drug for a receptor. Antagonist is the opposite: a non-physiological messenger which can bind the receptor, but this causes or the block of the response ore an inhibition of the response of the receptor. 


This communication is a multi-event communication: the activity must be manteined in the pre-synaptic element.

\chapter{Lesson 2: Receptors}
We can identify two big groups:
\begin{itemize}
\item{Intracellular receptors: with high probability \emph{cytosolic} or \emph{nuclear}. Actually, there are also ion channels presents in the cytoplasm that have a binding site for a ligand, like IP3 receptors, but since we are studying communication we can't consider IP3 like a first messenger. Intracellular receptors are just soluble molecules.}
\item{Cell surface receptors: the most commonly studied in neurobiology, they can be characterized by \emph{enzymatic activity} or \emph{coupled with G-proteins}, that are not ion channels. Ion channels belong to this category, but we will discuss about this later.}
\end{itemize}

\section{Intracellular receptors}
Cytosolic or nuclear. Usually, the ligands are hormones, so the ligand is lipophylic (or hydrophobic) and passes through the membrane, binding to the receptor in the cytoplasm or directly in the nucleus\footnote{In this case, the ligand must also traslocate through the nuclear pores}. This will take to the \emph{regulation of gene expression}, that is the biological response.

\subsection{Structure}
We can identify 5 domains in all such receptors, different in the number of amino acids. Each domain is responsible for different functions and they must all be there:
\begin{itemize}
\item{The A/B domains are involved in transcriptional activation: these receptors regulates the expression of a gene, so they can anchor the transcription machinery to the DNA}
\item{The C domain is rich in Cys, is characterized by many zinc-finger domains and such 3D organization allows the receptor to approach the DNA strand even in the condensed form, because there are macroscopic pockets that can bind it (major groove of the DNA)}
\item{The C-domain is necessary to approach the DNA strand}
\item{The D region is very small and is a inch region, flexible, because the receptor can change conformation in this region}
\item{The E and F domains are responsible as a all for the \emph{binding site} for the ligand}
\end{itemize}

Sometimes the same domains can operate a dimerization.
\\

After the formation of the receptor-ligand complex, the receptor can approach the DNA strand, but where? Each receptor has to modulate the expression of a precise gene! This is due to different HREs  (hormone-response elements) at the level of the DNA, different sequences that  recognize a specific  ligand-receptor complex. At this level, the DNA has different binding sites for different intracellular receptors. In HRE we always have palindromic sequences.

Why the receptors do not translocate into the nucleus in the absence of the ligand and approach the DNA, starting to promote the transcription of the gene? Because when the ligand is absent, the receptor potentially could bind the DNA, but is \emph{transcriptionally inactive}: that means that even by chance the receptor binds the DNA, it cannot start the transcription. When the ligand arrives, there are conformational changes in the receptor that will turn it in a \emph{transcriptional active state}.

In some cases, the ligand is the only actor needed to regulate this transition, bus sometimes there are also others cofactors needed! They are many time the heat-shock proteins HSP, like HSP90, HSP70 and HSP56\footnote{named on the molecular weight}.  Some receptor have in the C domain a sort of binding site that has affinity with HSP, so the receptor can be recognized by a ligand and by a HSP. When the ligand is absent, there is in  the target cell a complex formed by the receptor in inactive state, coupled with a HSP. This HSP arranges the receptor in a way in which the DNA binding site is hidden, and the receptor assumes probably  a globular form. In this form, the receptor cannot permeate across the nuclear pores. When the ligand arrives, his binding site is still available, even in the globular form: it binds to the receptor and this cause a change in the affinity of the receptor for the HSP, because the inch\footnote{regione a cerniera} region (D domain) assumes a linear form, and so the receptor \lfreccia the receptor is linear ant thiner than before, and now it can modulate the regulation of the gene expression ,because the DNA binding site is available and the linear form can permeate through the nuclear pores. This process is called \emph{transformation}.


\section{Cell surface receptors}
Two groups:
\begin{enumerate}
\item{Receptors with enzymatic activity}
\item{Receptors coupled with G-protein}
\end{enumerate}

\subsection{Receptor with enzymatic activity}
These proteins are receptors for hydrophilic ligand and enzyme with a catalytic region. The transduction machinery convert the extracellular ligand into an intracellular enzymatic activity. There are common features, even there are 60 families of receptors.
\begin{itemize}
\item{Localized in the plasmalemma}
\item{They are made of a single polypeptide chain which crosses the membrane only once.}
\end{itemize}
 We can identify the N-terminal part, extracellular, the region with the binding site for the ligand.  There is a small region, the transmembrane region of the receptor: this part of the molecule is an $\alpha$-helix. The third part is the C-terminal, which hosts the catalytic site (intracellular domain). When the ligand is present, the catalytic activity is activated.
 
 All of them are \emph{protein-kinases}. Protein can be phosphorylated in three aa residues: Ser, Thr and Tyr, the most numerous are those one in Tyr or Thr. The most common receptors ad tyr-kinase receptors. The ligands are usually or hormones or trophic/growth factors, important in manteining the survival in target cells.

When the ligand arrives \lfreccia \emph{dimerization} of the receptor. This is initiated by the ligand, so it never happens without the ligand. The crucial region for the dimerization of the receptor is the transmembrane region and the binding site. It's enough a point-mutation to avoid the dimerization\footnote{homo or heterodimerization}, even in the presence of the ligand. How the ligand causes this dimerization? If the ligand is a dimer, the first messenger has 2 regions that can be recognized by 2 receptors. In this case, the dimeric form of the ligand induces the dimerization of the receptors\footnote{of course we need also the right structure of the receptor}.
Not of the ligand are dimers: the second receptor recognize a different site of the ligand, but it's the first binding that causes the dimerization.

Dimerization is important to the activation of the transduction mechanisms: the first phosphorylation is the one of the two receptor, and the dimerization is necessary for this. This is called auto-phosphorylation in the books, but the right term is \emph{trans-phosphorylation}. The catalytic part of the enzyme becomes so phosphorylated. The catalytic part should be active even before the phosphorylation, consider that un recettore fosforila l'altro in quella parte e viceversa. The phosphorylation of the catalytic part enhances the enzymatic activity of the receptor. This happens close to the $\alpha$ regions of the receptor (iuxtamembrane region). There are also other tyr-residues in the C-term of the receptor, so the next step is the phosphorylation in other parts of the same receptor. That is the fully trans-phosphorylation. Trans-activation fully actives the enzymatic activity. So, also the C-terminal part of the receptor is phosphorylated.

(9 ottobre 2015)
Such catalytic region must be phosphorylated because: an intracellular protein binds to the catalytic site o the receptors, because it has to be phosphorylated, and such protein will be phosphorylated. Then the protein goes away. The second possibility is that proteins can interact to he phosphate groups at the level of the C-term and this allow a allosteric activation of those signal molecules. The third possibility is that such phosphorylated group anchor the signal molecules to the plasmalemma, and creates the condition for following phosphorylation (redistribution of signaling molecules). Like PLC, that can recognize these C-terminals and in this position becomes active (normally is not).
The interaction in the catalytic site or in the C-term needs an affinity between the potential binding site and the signaling molecules: this occur in the \emph{docking sites}, they are placed on the receptors and represent the part of the receptor that is phosphorylated\footnote{Before the phosphorylation, there is not docking site}. All the proteins with PTB or SH2 domains are able to interact with these sites. The specificity of the cell answer in guaranteed. These domains are important because if we find them in a purified protein, we can say that its receptor is characterized by enzymatic activity.

There are some proteins, called \emph{adapters}, that have a SH2 or PTB domain, but have no intrinsic activity, but works as an adapter to activate a protein with intrinsic activity without a PTB or SH2 domain. That means that if we purified a protein that as no PTB/SH2 domain, we cannot say that it has no intrinsic activity due to that same receptor. So, 1 receptor is enough to activate many signaling molecule, but this is not a loss of specificity in the cell answer, because the specificity between a protein and the receptor is guaranteed. 

At the endpoint, there is a modulation in gene expression. MAP-kinase cam be activated by such receptors.
\subsubsection{MuSK}
An example of TKR: Muscle-Specific Kinase. TKR means tyrosine kinase receptor.
It shows at least three levels of complexity. It has a single chain, an $\alpha$-elix at the level of plasmalemma. The ligand is \emph{agrin}: is quite big (200 KDa) and there are some isoforms. Neural agrin is secreted by motor neurons and is a neurotrophic factor. The secretion is activity dependent, so depends on the frequency of this charge: higher is the concentration in the synaptic cleft, higher is the activation of MuSK receptor in the muscle fiber. Neural agrin and MuSK control the structure of the end plate zone in the skeletal muscle fibers. The major endpoint is gene expression regulation, but we can control other processes: the receptor MuSK controls the distribution of a specific isoforms of colinergic receptors controlling its gene expression. The ligand can be dimeric or not, but we surely have the dimerization of the MuSK receptor, but MuSK doesn't work alone: in many cases such receptor do not consist in only one protein!  There are \emph{accessory molecules} that control the binding between ligand and receptor. We have MuSK receptor in post synaptic membrane in the neuromuscular junctions and such receptors are assembled with the Lrp4 protein: agrin  receptor is so a complex made by MuSK and Lrp4.

In the middle we have 2 molecules of Lrp4 and MuSK stay di fianco. Such receptor have a enzymatic activity at low level in the monomeric form. Many of those receptors are coupled with proteins to hide the catalytic site: Lrp4 block the catalytic activity of MuSK and, in this position between the two MuSK receptor, block a casual dimerization of MuSK that can occur in absence of ligand. Lrp4 is also involved in the binding: the ligand cannot binds the receptor if Lrp4 is absent. Lrp4 helps the building of the binding site and to start the signaling cascate. So, they are not so accessory proteins! 

When neural agrin arrives, it binds partially MuSK and Lrp4 (not sure about this) and the orientation of the parts change: musk in the middle and Lrp4 at the border \lfreccia full activation of the catalytic part of MuSK\todo{scarica il paper}.


The binding can  be really complicated, so the receptors need the accessory proteins to be activated. Since the docking sites are many, can activate a huge amount of molecules. This is an advantage, because we can activate many pathways with the same molecule.
So, there are three levels of complexity:
\begin{itemize}
\item{Binding site: dimerization and combination of MuSK and Lrp4}
\item{Many docking site}
\item{Ligand-indipendent activation of MuSK, because the density of the receptor control the dimerization of the receptor}
\end{itemize}

\paragraph{Regulation of nAChR isoforms and distribution}
When the cell is innervated, the AChR are localized in the end plate region, and the receptors are all the adult (junctional) nAChR, with 2 $\alpha$ subunit, a $\beta$, a $\delta$ and a $\epsilon$. High conductions and short opening time. Before the inenrvation, the receptors are re-distributed in the synaptic cleft because neural agrin is absent and there are embryonic receptors, characterized by $\gamma$ subunit.

These are 2 different isoform of the same receptors, so 2 different biological activities.

\paragraph{Ligand-indipendent way} The third level of complexity is that such receptors can be also activated in a ligand-indipendent way (it's still an hypothesis). before the innervation, the red signal is not homogeneous: we can distinguish some cluster of receptors. The nAChR is pre-patterned, so the distribution is not homogeneous and in some points of the skeletal muscle membrane and even in the absence of agrin, the AChR can form small clusters. If MuSK without Lrp4 has a partially active tyr-kinase activity, if we increase the probability of dimerization, the activation of the cascate is still possible. That's the ligand independent way, some MuSK molecules dimerize even in the absence of agrin. So, after the innervation Lrp4 is down-regulated, and also MuSK.

in the pre-patterned distribution, the cluster can be target for growth of motorneuron: when it is growing, it follows some signals in the extracellular mass and also in the membrane of the skeletal muscle. These clusters can fix the primary structure of the synapses in which the synapses can be established.

This is advantage when the biological response of this receptor is needed before the arrival of the ligand, like for fixing the synapses. After innervation and development have similar signals, that leads to the synaptogenesis.

\subsection{G-protein-coupled receptors}
Structurally,  7 transmembrane domains with 7 $\alpha$ helixes.  The intracellular region has no enzymatic activity, but it can interact with G-proteins (monomeric or trimeric), because they can interact with GTP and hydrolize it in GDP. In the trimeric forms, the $\alpha$ subunit is the catalytic part because has the binding site for GTP. The G-protein will interact with the \emph{effector}, intracellular that are or ion channels or enzymes. Many of these receptors are receptors for neurotransmitter, so they are called \emph{metabotropic receptors}\footnote{ionotropic if the neurotransmit binds to a ion channel}. So, there is a reaction between the extracellular and the intracellular domain: the extracellular signal, hydrophilic, is transformed into an intracellular signal, lipophilic.

Some neurotransmitters, like ACh, can be involved in ionotropic synapses, but in other synapses can activated a metabotropic receptor.
The G-proteins are involved in the metabotropic synapses. In the ionotropic transmitter, the neurotransmitter binds to a receptor that has a pore for ions.

(13-10-2015)

After the binding with the ligand, the receptor changes the conformation \lfreccia interaction with the G-protein. When the ligand is absent, the trimeric Gpprotein is anchored aat the plasmalemma, but the receptor has very low affinity for the G-protein: the protein is in the inactive form (3 subunit associated) and the $\alpha$ subunit il bounded to the GDP into the binding site. Receptor and G protein are inactive.

When ligand arrives, the conformational changes of the receptor increase the affinity or the receptor for the G protein, and the $\alpha$ subunit associates to the receptor \lfreccia dissociation of the G protein into $\alpha$ subunit and $\beta \gamma$, that caused a loose in affinity of the $\alpha$ for the GDP and an increase in affinity for GTP molecules. In such situation, the $\alpha$ subunit can interact with the effector: such interaction will last up to the hydrolisys of GTP into GDP, because $\alpha$ subunit is en enzyme that catalyze the dephosphorylation of GTP in GDP. If the ligand is still there, you can have a second event of activation of the same G protein; if the ligand disappears, the G protein will be inactive. We can consider the G protein like a time switch, that switch on the effector for a certain period.

The dissociation between the receptor and the G protein interferes with the receptor properties: we can consider the receptor in two form:
\begin{itemize}
\item{High affinity for ligand}
\item{Low affinity for ligand}
\end{itemize}

When the receptor is empty, has a very high affinity for the ligand, and the binding is favored. After the dissociation of the G protein from the receptor, this modify the affinity of the receptor for the ligand: the affinity of the receptor for the same ligand become lower \lfreccia this favors to switch off the mechanism of signal transduction. Of coure, the dissociation of the ligand will depends on the concentration of the receptor. This is an intrinsic negative feedback. If G protein remains longer associated with the receptor, the receptor has an abnormal affinity for the ligand. 

The $\beta \gamma$ has a role, or the $\alpha$ subunit is enough? This would be enough, because it has the hydrolytic action. The role of $\beta \gamma$: these subunits are presents in some hisoforms, so the same receptor could interact with different G proteins \lfreccia different subunits can have different affinity for the same receptor, so \emph{different time switch} for the effector! The other role of these subunits is to control the $\alpha$ subunit. The third role is that they could interact with specific effectors in parallel with $\alpha$ subunit: there could be a divergence or a helpful mechanism for the transudction.

\subsubsection{G-proteins modulate effectors}
Effectors can be:
\begin{itemize}
\item{Ion channels}
\item{Enzymes: }
\end{itemize}

The receptors are always anchored at the level of the plasmalemma, in the inner part.

\paragraph{Ion channels}: the end point is to change the permeability of the cell, so a \emph{change in excitability}. The interaction of the $\alpha$ subunit and the channel could interfere with the probability of opening of the channel. The permeability of the ion channel changes the membrane potential, that may cause a depolarization or a hyperpolarization. We can call this a \emph{metabotropic receptor}, because it causes the opening of an ion channel indirectly.

\paragraph{Enzyme}: the number of enzymes activated by G-protein is not so high, but there are at leat 3 enzymes:
\begin{itemize}
\item{Adenilil cyclase}
\item{PLC: phospholypase C}
\item{PLA: phospholypase A}
\end{itemize}
The reaction occur intracellularly and at least one product is soluble in the cytoplasm, so finally we have a messenger available at the level of the cytoplasm: this is a intracellular \emph{second messanger}, because its concentration is regulated by the concentration of a first messenger outside of the cell.

these soluble molecules are:
\begin{itemize}
\item{AC with ATP substrate, converts it in cAMP}
\item{PLC converts a phospholypid in IP3 and DAG. We have a change of Ca concentration, so Ca is a second messenger, even if is not produced by the effector. The real second messenger is IP3, Ca is a third messenger.}
\item{PLA  converts a group of phospholypids in arachinoid acid by hydrolysis (one leg of the lipid is broken down)}
\end{itemize}
 After the synthesis of these second messengers, that is an intermediate step, at the level of the cell in the cytosol different sensors for the second messenger are activated, and the biological respons depend on which sensor is activated. So, there are also intracellular effectors, not only the one on the membrane. Arachidonic acid is transformed into other molecules because of some enzymatic reactions, and the deriavatives activate the cell response. Receptors coupled with PLA are not used in neurobiology and not consider as metabotropic receptors.

The G-proteins can activate the effector, causing an increase of the second messenger concentration ,or inhibit the effector, causing a decrease of the concentration of the second messenger in the cytosol. These are equally significant for the biological response, activating or inhibit some reactions at the level of the target cell: in both cases, something changes in the target cell. This is why the trimeric G proteins can be \emph{inhibitory} or 	\emph{stimulatory}, $G_I$ or $G_S$.  A G-protein coupled receptor modulate the concentration of a second messenger, a decrease or an increase.

\paragraph{Sensors for the second messengers}
In mammalian animals are protein-kinases: depending of the level of a second messenger, we control the activity of protein-kinases, so the level of phosphorylation of intracellular proteins will change (increase or decrease if $G_I$ or $G_S$). If these proteins are a sensor for a second messenger, they must have a \emph{regulatory domain} for the binding, but since they are enzymes they must have also a \emph{catalytic domain} (may be more than one!!!). All the protein kinases are made by these two parts, they are enzymes regulated by the concentration of the second messenger.
\\

When a neuron transmitter binds to its receptor, it is probable that the level of the phosphorylation of proteins will change: the effectors can be AC or PLC in excitable cells, the second messenger is cAMP or IP3. The target proteins are responsible of the biologica response at the level of the target cell: that is a way to modulate cell phenomena, like speed up a chemical reaction or induce neurotransmitter release. The endpoint so is \emph{modulation of pre-existing cell activity}. This activity existed even before the arrival of the legand, is not about gene expression regulation this time, G-coupled receptors do not need to activate gene expression to activate the biological response of the cell. In parallel, we can also have gene expression, but in parallel!

In terms of communication, this means that the latency between the appareance of the ligand in the extracellular milieu and the response of the target cell is shorter, because we don't need to wait gene expression and translation. This is way we'll NEVER find an intracellular receptor or a receptor with enzymatic activity in post-synaptic that activates gene expression, it will take too much time and we need a fast cell response.

\paragraph{Metabotropic receptors} every receptors coupled with G-proteins. In this case, the post-synaptic signal will start with the opening of an ion channel ,but the channel is not intrinsic with the molecule that has the binding site for the neurotransmitter. The activity of the ion channel can be modulated:
\begin{itemize} 
\item{via a G-protein: the simplest way}
\item{via a second messenger: activation of the G-protein \lfreccia activation of enzyme \lfreccia second messenger \lfreccia ion channel.}
\end{itemize}

Ionotropic receptors will be seen int he third part of the course.

The first type of metabotropic synapses (via G-protein) have less longer latency that the second type (via second messenger): this is the slowest synapse. If we compare all this communication system to other receptors, this communication is also faster than cellular communication via intracellular receptors or receptors with enzymatic activity, because here we don't need gene expression!

In most of the cases, second messenger induces cell response via protein-kinases. Another possibility is that it activates directly a molecule that is not a protein kinase: there are many of these proteins, they have a regulatory domain in which the second messenger could bind, but these proteins are not enzymes. Protein EPAC can modulate their activity by cAMP, but they are not kinases: one target cell could answer to a ligand or with a kinase or with proteins like EPACS, it depends on how the signal transduction works, because these two type of proteins have a different affinity for cAMP: PKAs have a higher activity for cAMP, EPACs have a lower affinity, so the via will depend on the concentration of the second messenger. If we have a prolonged exposure of the cell to the same ligand, there will be higher levels of cAMP, so the ligand can activates PKAs and EPACs, not only PKA.
A change in the level of the second messenger is enough to lead to a different biological response in the target cell.

The cell answer will depend also of the time and space presence of the ligand.



\subsection{Advantage of the multi-step signaling cascade}
(15-10-2015)

The main advantage is that guarantees the \emph{amplification} of the extracellular signal. In theory, one molecule of first messenger is enough to induce the phosphorylation of millions of proteins. It can activate several G-proteins, if the ligand remains in the synaptic cleft for like 10 ms. The G-protein can activate a lot of effectors., and each of those molecules (AC od PLC) catalyze the production of many many second messengers. More the steps, more the possibility to amplify. That is the reason why in the presynaptic terminal there are only few vesicles. 

Is there amplification if we have an intracellular receptor or a surface receptor with enzymatic activity? Yes, because the aim is the gene expression, it is just less evident. 

When 1 ligand is enough to phosphorylate many proteins, if it appears is a unregulated way, the post-synaptic elements give the response even if the signal is wrong, so the amplification is ok when the communication in physiological. It is more complicated for the target cell to switch off cell communication, because of the amplification. 
The fact that many step exists guarantee an efficient switch off of the signaling cascate: intrinsic negative regulation in each step. Such automatisms protect the target cell from over-stimulation. At the level of the cell membrane of the post-synaptic element there are mechanism that avoid the overstimulation:
\begin{itemize}
\item{Regulation of the concentration of the first messenger}
\item{Regulation of the number of the available receptors}
\end{itemize}

Even if the ligand rests in the synaptic cleft for long time, the post-synaptic membrane can avoid damages. For concentration of ligand, in a synaptic cleft the space is not very big and outside the cleft the concentration of the ligand is 0\footnote{exocytosis of the neurotransmitter occurs only in the synaptic cleft of course}, so the \emph{diffusion} of the neurotransmitter from the cleft to the external space contributes to removal of the neurotransmitter. This contribution is significant when the secretion of neurotransmitter stops: the diffusion promotes the decrease of its concentration in the cleft.  The second way to control the level of neurotransmitter is the \emph{re-uptake}: the pre-synaptic terminal after secretion re-use the ligand still available in the synaptic cleft to reload. This is possible because there are mechanism to transport it into the end terminal of the nerve, consider also that the making of the neurotransmitter requires a lot of energy from the cell. The third way is the \emph{enzymatic degradation}, of hydrolysis of the neurotransmitter: each molecules of neurotransmitter has a half-time of life, life for ACh in a colinergic synapse, so if we wait enough, each molecule will be degradate or re-uptaken by the presynaptic element.

There is another mechanism that involves the control of the number of available receptor\footnote{if the first mechanism fails}. This mechanism is called \emph{desensitization}, and there are 2 type: 
\begin{itemize}
\item{Rapid}
\item{Slow}
\end{itemize}

The rapid desensitization is based on the phosphorylation of the receptor involved in the response: this starts the fast desensitization of the post-synaptic element. So, the receptor is still present, but its function is compromised: the phosphorylation inactivate the receptor from a functional point of view.

The slow desensitization is based on the internalization by endocytosis of the receptor. In this case, the receptors are not available anymore, so its density decreases.

When everything is ok, we have the effector pathway. 

In case of a prolonged exposure of excess of the ligand, there will be a rapid desensitization: if the complex ligand-receptors lasts longer the usual, phosphorylation in the intracellular domain of the G-coupled receptor inhibits the activation of the G-protein by the receptor because it blocks the interaction between these two elements. 

If the ligand is removed somehow, the phosphorylated group of the receptor are hydrolised and everything is normal. 

If the exposure is very very prolonged, the transition occurs into slow desensitization: the receptor remains phosphorylated longer and this increases the probability of an association between the phosphorylated receptor ad an intracellular protein, the $\beta$-arrestine. $\beta$-arrestine is always there, but the probability of association in a matter of time\footnote{The first phosphorylation of the receptor blocks the association with the G protein; a prolonged presence of the ligand let the receptor to be more phosphorylated and this new phosphorylation lead the binding of the $\beta$ arrestine}. When this occurs, such receptors are internalized via clatrin-coated way. When the receptor is internalized, we have 2 possibilities depending of the cell:
\begin{itemize}
\item{The intracellular receptor is degradated}
\item{The receptor remains internalized and exposed again when there is no more risk of over stimulation (faster): this process is called \emph{resensitization}, the opposite of slow desensitization.}
\end{itemize}

The signaling cascade is also strictly dependent on the ligand presence. Each step of the cascade is reversible, and to maintain it active we need the ligand outside. The G-protein is continuously activated when the ligand is present, then the hydrolysis of the GTP into GDP cause the inactivation of the G-protein, so it is important the \emph{time activation of G-proteins}. Also the \emph{activation time of effectors}: to maintain the enzyme active is required the presence f the ligand. 

There is also an intrinsec control: also \emph{second messengers have a half time of life}, so to maintain the signaling cascate we need the continuous presence of the ligand to keep producing the second messenger. When the ligand disappears, the concentration of the second messenger will go down (or up if the G-protein is inhibitory).

Also the \emph{half life of phosphorylated proteins} is important: this is reversible due to the presene of kinases and phosphatases\footnote{Phosphatase prevails when the ligand is no more there, g-proteins are inactive and so do the effectors}. 

\chapter{Calcium signaling}
Calcium signals have different meanings in the cells: some represents a starting point for cell death (apopthosis, irreversible) and some others encodes for physiological processes (reversible). This means that these signals are not equals: there are specific properties different for each signal.

An answer to this bivalent meaning of the calcium can be the activation of different \emph{sensors} in these two situations. 

There are protein kinases which require calmodulin to be activated, so we can distinguish 2 families of kinases:
\begin{itemize}
\item{Protein kinases C: in the cytoplasm, bind Ca}
\item{CAM kinases: Ca calmodulin-dependent}
\end{itemize}

These two sensors have 2 domains: the catalytic domain and the regulator domain to respond to a specific second messenger. The regulatory domain of CAM kinases bind calmoduline even whet it is saturated of Ca: because of this saturation, such protein kinases will be activated. Calmoduline saturated is needed for this activation.

In protein kinases C there is a direct binding of Ca with the regulatory domain. To have the full activation, these proteins must be anchored to the cell membrane and this step of anchoring is DAG-dependent\footnote{Diacyl glicerole}. In most cases this translocation in Ca-dependent, but this is not the rule.

(16-10-2015)

There are also other sensors of Ca: they are proteins that can bind Ca and start immediately the biological response (for example, this favors the exocytosis of the neurotransmitter in the synaptic cleft). 

Which is the different between a Ca signal of cell death or a more physiological responses? 

In pathological or extreme conditions, when the apopthotic process is needed, there is a higher concentration of Ca and/or longer duration of this higher concentration. To activate the Ca sensors, we need a huge amount of Ca and a prolonged release of Ca in the cytoplasm. These sensors have a low affinity for Ca, that's why we need a high concentration of Ca.

In physiological conditions different to cell death, the Ca signals are more transient and the level of Ca is lower. Measuring in time the variation of Ca in the cytoplasm, the shape of the profile is a pick and then a hyperbole. At the very beginning , the concentration is 100 nM. When cell communication requires Ca, a very fast increase in concentration occurs, so we reach a pick (\emph{raising face of the Ca transient}) and then the decay phase starts and reach the basal concentration of Ca. The peak is slower in terms of value if compared to the Ca levels reached just before the apopthotic process and the duration of the transient is shorter. So, the cell control rigorously these processes. The shape of this graph is similar to the action potential, but it least longer (50 ms against 1-2 ms of the action potential). 

In the cell, there are mechanism which allow the increase in Ca concentration and mechanism that take back to the basal level. We can distinguish in a cell the \emph{ON-MECHANISM} that sustain he raising of Ca, and \emph{OFF mechanism} responsible for the decay of Ca transient.

ON-mechanism in red and OFF in green:  between the intracellular compartment and the extracellular environment, there is a big difference of Ca concentration (inside: 100 nM; outside: 1,5-2 mM). There is also an electrical gradient: the negative side in within the cell, so we have a gradient for the entry of Ca that increases the inner concentration, increasing the permeability with ion channels.
\section{ON mechanisms}
The mechanism to have the rising phase of the potentials (ON) are:
\begin{itemize}
\item{VOC: voltage operated Ca channels. In front of a change in the membrane potentials, such channels open and, since they are permeable for Ca, the Ca will enter into the cell cytoplasm, giving rise to the rising phase of the Ca signal.}
\item{GOC: G-protein operated Ca channels. This could change the probability of opening of ion channels.}
\item{SMOC: second messenger operated Ca channels. Some Ca channels can be controlled in their activity because the phosphorylation vary the activity. Second messenger may interact with ion channels causing their opening. There a re K channels which are dependent of the concentration of Ca inside the cell.}
\item{ROC: receptor operated Ca channel, or ligand-dependent. There is a binding site for a ligand: the ligand increase the open probability of these receptors, that internalize Ca (ex nicotine-ACh receptors).}
\item{SDOC: store-dependent operated channels. The mechanism of activation requires interaction between intracellular compartments and the extracellular milieu. Stores are intracellular compartment that can accumulate Ca: depending of the level of Ca reached in these stores, the probability of openings of SDOCs change: if a low concentration, the probability of opening increase to favors the influx of Ca.}
\end{itemize}

We have also receptors coupled with G-proteins: it can interacts with PLC and IP3 is the second messenger that increase the level of Ca inducing the release of Ca from the intracellular organelles (stores). so this is independent from the electrochemical gradient. The mechanism is efficient, independent from Ca concentration outside and gradient, because depends exclusively on the presence of the ligand (for example, a neurotransmitter) outside.

To distinguish from a transient due to an influx of Ca in the cell or a transient that involves the release of Ca from stores, we just put a cell in a milieu without extracellular Ca. 

There are also intracellular ion channels, localized at the level of the membrane of intracellular organelles (RE, nucleus etc):
\begin{itemize}
\item{A channel activated by IP3: this release requires an extracellular signal (ROC)}
\item{A channel that can be found near the channel activated by IP3 or on another organelle. Such ion channels open when Ca concentration increases in the cell, so they requires Ca to release Ca. These channels are known as \emph{Caffeine-sensitive receptors} or ryanodine receptors: they were discovered when some scientist studied the effect of caffeine and ryanodine in the cells. Ryanodine is a ligand, a molecule present in tropical plants. This is a receptor named by its exogenous ligand (the endogenous ligand is Ca).} 
\end{itemize}

\subsubsection{IP3 receptors}
These two channels are tetramers and, in case of the IP3 receptor, each subunit has a binding site for IP3, so the channels open when the subunits are saturated (son we need 4 IP3 molecules to open the channel). The effect is the exit of calcium from the lumen of the organelle to the cytoplasm \lfreccia the rising phase of the transient starts.
When there is no IP3 (or the level goes down), the Ca release stops. The IP3 receptor is sensitive to the Ca intracellular concentration, so Ca can modulate the probability of opening when IP3 is present\footnote{So IP3 allows the opening, Ca modulates only the opening probability, that's why this channel is IP3 dependent and Ca is a co-factor}.
There is a Ca dependency on a bell-shaped curve.  When Ca concentration il low, the probability of the channel is low; maintain [IP3] stable and increasing Ca, the probability increases. When the Ca increase a lot, it has a negative control on the probability of opening of the receptor, even if IP3 is present. The right part of the curve avoids an over-stimulation of Ca release from intracellular store. This is a sort of negative feedback. 
 
 The binding site for Ca is different from the binding site for IP3, so the Ca interferes with the conformation of the channel. There are 3 hisoformes of the IP3 receptors:
 \begin{itemize}
 \item{Type 1}
 \item{Type 2}
 \item{Type 3}
 \end{itemize}
 
 Under a functional point of view, these hisoforms have a different affinity for Ca and IP3: in a cell they are segregated in different compartments and depending on the level of Ca and IP3, we can reach different hisoforms. 

\subsubsection{Ryanodine/caffeine sensitive receptor}
Tetramers, there are 3 isoform. The type 1 is present in skeletal muscle fibers; the type 2 is present in the cardiac muscle cells; type 3 is distributed in different cell tissues. 
Each subunit has a binding site for Ca, ryanodine or caffeine (the binding sites are different). So we need 4 ligands for 1 receptor to open it, Ca\footnote{Maybe the modulation starts even when the receptor is not saturated} or ryanodine or caffeine. Also in this case there is a Ca dependency: Ca represent the ligand for these receptors. A bell-shaped curve, to avoid the over release (negative feedback), so Ca controls its own release from this intracellular compartment. The pick is in the range of $\mu M$ concentration of Ca. When the IP3 is not activated, the ryanodine receptor is active (this in the $\mu M$ level of concentration). When the IP3 goes up, the first event is the release of Ca: Ca starts to increase and when in the micromolar range, the IP3 receptor closure is favored, while is favored the opening of the Ry-caffeine receptor. This is why the intracellular calcium stores sensitive to Ry or caffeine are called \emph{amplifiers}. Everything depend on the Ca release in the first step, that is governed by the IP3. This mechanism of release is called SICR (Ca-induced Ca release). 


Let assume that another ligand increase the IP3 and the release of Ca: the amplifiers, if you are below the $\mu M$ phase, will not contribute to the rising phase of Ca signal. These calcium stores can be amplified by IP3-dependent Ca stores, but also from Ca in the $\mu M$ range, when such range is reached via the mechanisms at the level of the membrane, like via VOC. The only thing that they requires for amplify the signal is the correct range of cytosolic Ca ($\mu M$). Remember that ryanodine and caffeine are liposoluble and enter into the cells. 

(20-10-2015)

NAADP (nicotine acid adenine dinucleotide phosphate) is the $3^{rd}$ ligand that triggers the release of Ca from intracellular compartment. If NAADP in a cell, we can increase Ca ina way that is independent on the concentration on Ca outside: NAADP so is not an aactivator of a channel on the cell surface, but induces a release from intracellular compartment. This is not demonstrated in all cell.
\section{Intracellular Ca stores}
The problem is the localization of intracellular Ca stores: the localization of these deposits varyes if you consider different cells, but Ca can be store in RE or sarcoplasmic reticulum, in the nucleus at the level of the nuclear envelope, where there are IP3 receptors, in some parts of the Golgi apparatus, in secretory vesicles that release Ca in a regulated way, and also in favor of the existence of specific organelles that have the only role to accumulate Ca, the \emph{calciosomes}. 

The localization of Ca in organelles is not so important because the idea is that such deposites can be rearranged depending on the stimulation of the cell: the content of intracellular Ca is not a constant, but depends on the stimulation of the cell: the Ca stores are dynamic, there is a plasticity of such organelles that can enlarge or decrease, depending on the extracellular request.

Organelles in which Ca can be release by IP3 are confined around the nucleus. The green signal is different in time: that's because the IP3 receptors change places and are characterized by a wide distribution. More organelles can respond to the IP3 signal. The Ca signal is proportional to the intensity of stimulation. 

\subsection{Ca-OFF mechanisms}
they are important for the reversible funcitonal answers
Some are localized in the cell membrane and others are intracellular:
\begin{itemize}
\item{Exchanger that comes out Ca using the chemical gradient of Na. It's a passive mechanism that uses the electochemical gradient that causes an influx of 3 Na in the cell. This exchanger can operate in opposite direction depending on the concentration of Na and Ca \lfreccia recovery of Ca concentration after the generation of the signal}
\item{PMCA pump: plasma membrane Ca ATPases, they require energy to extrude Ca. ATP is the source of energy: these pumps are very efficient. It can be blocked pharmacologically}
\item{SERCA: sarco/endoplasmic reticulum Ca ATPases. They are Ca pumps intracellularly localized: uses ATP as energy source to transport Ca from the cytoplasm into the lumen of intracellular Ca stores. They have 2 roles:
\begin{itemize}
\item{Contribute to the decrease of Ca after the pick of the Ca response: via such pumps, the basal level of Ca is reached}
\item{They also are necessary for the reloading of Ca stores after a release event.}
\end{itemize}
If a calciosome is presente, since it is a storage for Ca, has to express on the membrane a channel and a Ca pump.}
\item{Exchanger at the external level of mitochondria. The role of mitochondria depends on the cell that you consider: in some cells they store Ca only when the pick of concentration is very very high; if it is low, they probably don't have a specific role of reload \lfreccia they are not always needed. In some cells, very high level of Ca is reached very often, like in muscle cells: there mitochondria are very important to recover the basal level of Ca. The affinity of these mechanisms is low so a huge amount of Ca is required. These mechanisms partially involve exchangers, but many are not known yet.}
\end{itemize}

An organelle can store ca if it has a channel for Ca and a pump to reload Ca after a release event!

The pharmacology between PMCA (on the memrbane) and SERCA (intracellular) is different: experimentally we can inhibit one pump without touching the efficency of the other. The Ca moves to be redistributing in the cell. If we quantify Ca, the quantity of Ca present in the cell remains the same, it just change place (from the organelles to the cytosol and vice versa).

The affinity of PMCA and SERCA for Ca is different: SERCA has a higher affinity that PMCA, because the most important thing is to reload stores to reach the basal level of Ca after the signal. That's because the prolonged presence of Ca in the cytoplasm is a risk, a non physiological condition.

In the mitochondria, the accumulation on Ca is transient: they don't release CA in a regulated way, because in the lumen there is a high concentration of phosphate groups (see oxydative phosphorylation). When the concentration of Ca in the cytoplasm reaches very high levels, that's a dangerous condition and the cells uses al the mechanism to remove it, also the mitochondria\footnote{Let's say that they are a safety mechanism}, but Ca cannot remain there for long time \lfreccia Ca goes into them and then other mechanisms, very slowly but continuously release Ca in the cytoplasm. The Ca which enters back in the cytoplasm is not a significant signal to activate the cellular response.

In any cell, the Ca level is an equilibrium between OFF and ON mechanism: when the Ca signal is triggered, there is  prevalence of the ON mechanism \lfreccia the Ca level in the cytoplasm increses \lfreccia pick reached \lfreccia reload: prevalence of the OFF mechanisms. Some of these mechanisms work at the basal level when the signal is not there: sometimes one of the ON mechanism is disregulated, so the level of Ca change! This happens during \emph{aging}: in all the cells there is a progressive increase of the basal level of Ca than in the younger part. Also the signal changes: the pick can be higher or lower, the duration of the Ca signal longer or shorter. Aging influences these mechanisms at rest and when generating a signal: some processes are enhanced and some silenced, but the rule is that \emph{the basal level of Ca is higher} in excitable and non excitable cells.

\subsection{Ca handling}
If we consider the Ca sensors in a cell, some have no apparent functional activity: they have binding sites for Ca but no functional activity \lfreccia \emph{calcium buffering proteins} or Ca buffers. They are S100, calreticulin, parvalbumin, calbindin and calsequestrin. What is their functional meaning? Some of them are in the cytoplasm, some within the lumen of the stores organelles. 

The one present within the intracellular Ca stores are important because the are sequestring Ca, increase the ability of the store to accumulate CA, avoiding the precipitation of the ion\footnote{Ca is not so soluble}. They are very big molecules with a large number of binding sites for Ca. We have 3 elements which characterize each Ca store.
\begin{itemize}
\item{Buffers inside to accumulate Ca without precipitation}
\item{Pumps at the level of the membrane to allow the reloading of Ca}
\item{Channels for the release of Ca in the lumer}
\end{itemize}

You have to demonstrate 2 of these points if you find a new organelle that could be a store of Ca.

Those buffers are diffusible, so such molecules are part of the cytoplasm, that is a sort of gel. Because of their binding sites, they manage to control the diffusion of Ca within the cytoplasm. Due to the fact that these proteins are present on the cytosol, the diffusion on CA in the cytosol is not efficient, not a long-range mechanism \lfreccia Ca is a \emph{short range messenger} in the cytoplasm.

There are Ca buffers that are real competitors for the ON and the OFF mechanisms: they can inhibit the functional activity of Ca! Someone measured the diffusion coefficient of Ca in the cytosol and compared it with the one of Ca in a solution without Ca buffering proteins: this is 2223 $\mu m^2$/s, but in a solution with buffering proteins this is 13! This means that in this milieu the free-Ca is hidden by ON or OFF or buffering protein.  

IP3 has a diffusion coefficient very very high, even if bigger in terms of molecular weight.

This short range of action of Ca has functional results:  in a very complex cells, like a neuron, when Ca increases, the spatial organization of Ca depends on the localization of ON mechanisms. Ca will increase just close to its origin, so the Ca buffering proteins confine the Ca in macro-domains. A dendrite is characterized by lots of spine: Ca can increase in one spine and not in the other or in the dendrite. You can have macrodomains of Ca even in a round cell: it's enough that the diameter of the cell is longer than 10 microns.

In the same cell, many processes are Ca dependent: how to discriminate between them? such Ca bffering proteins increase the degree of freedom for Ca signals, in terms of space and specificity of the cell answer. Such Ca buffering proteins are also invlved in the kinetics of the Ca signal: they regulate the duration of the Ca signals, because they bound Ca and are competitors for ON mechanisms. Increasing the potential of these proteins may let us have a lower pick. 

To decode the different signals generated by different inputs:
\begin{itemize}
\item{We need to know the concentration of Ca reached, for trigger the specific cell answer}
\item{Know the spatial organization of Ca signals}
\item{Know the temporal organization of Ca signals: how long the ligand increases the concentration of Ca? Some of these signals are trigger of cell death!}
\end{itemize}

(22-10-2015)

\section{Fluorescent Ca dyes}

They work like fluorescent antibodies, but they are specific for Ca. We have to use appropriate light to excite such molecules. If before their discovery you can just quantify how much Ca there was in a cell, now in different color we can see different Ca concentration in different areas of the cell. Apart the pick of kinetics of the transient, we can recognize the spatial localization. The colors used are \emph{pseudocolors} because such Ca indicators emits just 1 color, but depending on the variation of the concentration the intensity of the color changes. To discriminate better from changes, a computer program associates a scale of pseudocolor to the intensities. White means saturation, very high levels of fluorescence; blue means very low concentration.

When Ca increases, if we  label the cell with Ca indicators, the monochromatic fluorescence increase in certain areas. Such cells are \emph{still alive}: we can record the answer generated by a ligand. The fluorescent Ca dyes guarantee to transform the B/W curve of Ca signal into colors. Such dyes were applied in skeletal muscles: when these cells exhibit contraction, we can record what happens in the same cell measuring Ca. We can describe elementary Ca events in cardiac cells: Ca can increase in discrete areas of the cell \lfreccia evidence of the implication of low coefficient diffusion of Ca. 

If one ligand causes a Ca signal which is localized in a certain area of the cell, what happens when Ca has to increase in the entire cell? This is due to an active propagation of the Ca signal, because diffusion is not efficient. This is quite similar to action potential: if we need a general Ca signal, there are active mechanisms that guarantee the propagation of Ca far away from the vey beginning of the Ca signal. Such propagation mechanisms are named \emph{Ca waves}. This phenomenon can be associated with metabolism, contraction, exocytosis etc, and they are based on the regular distribution of the ON mechanisms in the cell. The hypotesis is that the ON mechanisms generating the Ca signal are superficial, like a mechanism that leads to the increase of IP3: IP3 induces the release of Ca from IP3 Ca stores, and that's the beginning of the signal. If all the ON mechanism are near the plasmalemma, because of the low coefficient of Ca, Ca increase only near the plasmalemma (because of buffer Ca proteins). If the on mechanism are regularly distributed in the entire cell, they cover the distance between Ca sensors near the plasmalemma and sensor far away: an increase of IP3 concentration leads to the release of Ca close to the trigger zone, than diffusion of the Ca to the second level\footnote{a bit more fare away from the plasmalemma} of ON mechanism within its short rage action distance, where the Ca signal is regenerated, and so on. So, we have the \emph{wave front}.

Such events are called \emph{Ca waves} because we can identify and measure the wave front by microscopy: administrating BK or ATP let the wave go in opposite direction (BK from dendrites to soma, ATP from the soma to dendrites). One ligand induces a specific response in the cell because Ca increases in certain zones; even when the Ca propagates, the specificity of the cell answer can be encoded by the direction of the propagation. If Ca is propagated from dendrites to the soma, probably the signal is the activation of something. So, the same signal with different spatial organization and orientation will encode different cell answers.


There are also \emph{temporal aspects} of the Ca signaling: in a non excitable cells, when stimulating it three times, one with low concentration of a certain agonist, then with higher concentration, and then still a higher concentration, assuming that such ligand induces a Ca signal in the cell, we can see that the ligand induces \emph{repetitive Ca signal}. In any case, we have repetitive and consecutive Ca transient: this is called \emph{Ca oscillations}, because we have Ca at basa level, then up-down-up-down and so on. The difference between these 3 events is the frequency: a higher concentration of agonist generate a more frequent Ca signal. The cell generates Ca oscillation characterized by different frequencies and each frequency will trigger a specific cell answer: in this way we increase the degree of freedom of the system.

Neurons, during the developments, have a common Ca oscillations behavior, even if there are still no contacts between nervous system. There are 2 patterns, one in the soma and one in the growth cone: these two patterns can be identified an measure by these methods \lfreccia low frequency in the soma, higher frequency in the nerve ending: these 2 patterns occur at the same time in the cell, but they are enough far away that there are independent variation of Ca. They menage to reduce the freqency in the nerve ending: they discover that the growth of the cone immediately stops., so this signal encode for the growth of the neurite\lfreccia demonstration that each frequency encodes for a different cell answer. 

At the level of the soma, they tried to alter the frequency: this dis-regulate the gene expression of the neuron. A neuron committed to secrete ACh was transformed in a neuron that secreted another neurotransmitter.


Sometimes Ca oscillations occur spontaneously (\emph{spontaneous Ca oscillations}), like during the development: the frequency is important to determinate the fate of the cells and the frequency that occur at the level of the elongating neurite is important to modulate such elongation. Interfering with excitability of cells during the development, like increasing the gene expression of some K channels in the nervous system, Ca oscillations were different, because increasing the permeability of the membrane for K, K goes out of the cells \lfreccia hyperpolarization, the cell elements becomes less excitable. Let the membrane potential -80 mV and the threshold for the action potential -60mV, and let's consider a cell with the same threshold, but -90mV of membrane potential: \emph{the gap  describes the excitability of te cell}. Let's assume that in a cell in which membrane potential is -80mV, we enhance the production of K channels: the probability to have K channels opens will be increased. If they open, the K goes out from the cell and the membrane potential at rest is lower than usual. Looking at the distribution of the neuron, like labellinfg them, at the level of the spinal chord, these author realized that the Ca oscillation interference manage to regulate the number of each type of neurons: this was the first evidence in an animal that Ca oscillations encode for the gene expression of the genes responsible for the production of the neurotransmitters. 

Such Ca oscillations are not peculiar in neurons: if we thin that Ca signals can regulate gene expression, all the cells during differentiation need a regulation program. In excitable elements, we have 2 mechanisms of Ca oscillation:
\begin{itemize}
\item{During development, there are \emph{intrinsic Ca oscillations}: such oscillations are associated to spontaneous action potential, so before synaptogenesis neurons are able to generate in autonomous cells some action potential (like in cardiac cells). This happens in neurons, skeletal muscle cell, cardiac cell. During differentiation, Ca is one of the messenger that regulate gene expression: as neurons as skeletal muscle fibers can generate spontaneous action potentials even in the absence of the signal. If this is true, even if we consider an adult neuron and force the neuron to have an unusual electrical activity, we can force the neuron to generate unusual Ca oscillation}
\item{In adulthood, we can modify the gene expression profile (potentially) and transform a neuron from colinergic to serotoninergic, even if the neuron is terminally differentiated. SO, a neuron is differentiated only because it receives certain frequencies of stimulation, so the Ca oscillations are definite, and so the gene expression profile.This  happens during plasticity, and the modulation of the frequency in this case is not significant, but we can take neurons and modify them even in adulthood changing the frequency received by this neuron.  So, the frequency depends on network electrical properties. Experiments of transplantation of peripheral ganglia into CNS are already begun.}
\end{itemize}

(27-10-2015)

\paragraph{How the cell can count the frequency and identify different signals on frequency?}  We have to identify a molecule that counts the events in terms of time. There is 1 molecule that is able to count and define the frequency of oscillatory activity, and this is a particular hysoform, \emph{CAM kinase II}: CAM kinases catalyse the phosphorylation of proteins, they are modulated by CA in an indirect way, because they can bind Calmodulin, that can bind Ca.  One molecule is a complex of molecules: this hisoform operate with other calmodulin molecules. In each molecule, there are a regulatory domain, a catalytic domain and a domain do aggragate different calmodulin molecules (all CAM-kinases II). There are 11 members in the complex that are normally aggregates.

We have many binding sites for the Ca. Each member has a regulatory domain, the black dots is Calmodulin: when saturated with Ca, has an affinity for the regulatory domain of each member, after the binding the catalytic domain is activated and the phosphorylation occurs on the closer member of the same aggregate, and it is activated as well.

Considering only 1 member, when calmodulin is not saturared, the regulatory domain hides the catalytic domain. Then Ca level goes up \lfreccia the probebility to have Calmoduli saturated increases: calmodulin binds the regulatory domain \lfreccia conformational change of 1 member of the complex: there is a transphosphorylation of close members if 2 members are activated at the same time.  We need to have 2 members close and they both need to be phosphorylated, because the opening of the molecule is revesible,a and the kinetic of this opening is regulated by phosphorylation: each members rest active for a long time after phosphorylation, even if it is non required to activation of catalytic domain (that is due to the arrival of Calmodulin). The catalytic domain phosphorylate the regulatory domain of the close member.

When only some member are activated and phosphorylated, phosphorylation is  not possible  because not high enough to trigger the biological response: the complete activation of each member is controlled by the frequency of Ca events. That's because there is not so much Calmoduline saturated. If Ca goes down because the frequency of the events is not high, we have the reversibility of the activation, a intramolecular phosphorylation that do not come at the other molecules and the aggregate inactivates. If the frequency is high enough, we have before the revesibility, another Ca event. The second spike increases the probability that more members close to each other are trans-phosphorylated. There is a threshold. This is demostrated for the hysoform type 2, but other hysoformes can sense different sequences. \emph{Different hysoforms respond  to different frequencies}. Also playing with the number of members of the aggregate is involved in the different responses: higher number, probablly higher frequency required.

In a column perfused with a solution in which the author controlled the Ca content, it has been isolated  CAM kinases: swithcing from low Ca and high Ca, they identify a certain frequency at which all the molecules were activated. hHe activation of the molecules fixed in the column was the product of the reaction: they purify the phosphorylated substrates and quantify them, and they made an association \lfreccia high frequency, lots of substrate phosphorylates. So,  a molecule can be activated and the efficiency of activation depend on the frequency of the variation of Ca in a solution.

\chapter{cAMP}
\section{cAMP in cardiac cells}
As well as for Ca, in the same cell cAMP can be increased, but different responses can occur. How the same molecule encodes for different cAMP signals?

In a cardiac muscle cell, increasing the level of cAMP can modulate some aspects of the metabolism (like glycogenolysis), of the contractility and gene expression. There are different sensors for cAMP, like PKAse, EPACs, that have different affinity for cAMP. Also for cAMP can exist a modulation in terms f localization of cAMP increases and production and spatial timing of cAMP signals

There are no sensors that emits florescence, so we cannot do the same thing as Ca with fluorescent dyes. All the methods are indirect, but since Ca is a second messenger regulated in terms of concentration, time and space, the same probably occurs of the cAMP: it reaches different levels, is localized in different manners and the signals are also different in terms of kinases, but nobody have seen some cAMP oscillation.

Indirect evidences: if we consider a cardiac cell and use antibodies to identify the $\alpha$ subunit of a g-protein close to adenil cyclase (AC), we identify a discrete distrbution of g $\alpha$ subunits. Such pattern of distribution is in stripes, and probably it has a functional meaning. If we immunolabel the AC, the distribution is again not homogeneous, is a discrete distribution. Another evidence is that each cell has a gene expression profile for the \emph{AKAPs}, the A-kinase anchoring proteins family: such molecules anchor PKAs in discrete zones of the cell, so PKAse are not diffusible in the cytoplasm, but they are anchored to membranes of specific organelles, and each cell has many different AKAPs. Each member is classified on the basis of the molecular way, like AKAP100. Overlapping the image of the isoform of PKAse and AKAP100 shows that a particular hisoform of PKA is linked to a particular AKAP.
This means that all the members involved in cAMP signaling are blocked in discrete area of the cell. The cAMP signaling and the activation of PKAse occur only in some region of the cell, but not because the cAMP is not diffusible, but because the actors of the signaling are fixed in these region. This guarantees the activation of different processes in different places of the cell.


(29-10-2015)
\section{FRET}
An innovative experimental approach takes advantage of the existence of non-toxic molecules that emits fluorescence in living cells. This consist in an engineering PKA, to coniugate yellow fluorescence protein and channel fluorescence protein in a PKA molecule. In this way, they synthesized a fluorescent PKA. The conjugation was made in a specific way: they decide to coniugate the ciano fluorescent protein (CFL) with both regulatory domain of PKA, and yellow fluorescence protein to (YFP)  the catalytic subunits. Potentially, the PKA could emit fluorescence intto 2 different range of colors: yellow and ciano.  In this configuration, when PKA is inactive and the regulatory domains are associated to the catalytic domains, we can take advantage of the FRET phenomenon (fluorescence resonance energy transfer). When we have 2 fluorescent molecules very close, and if the spectra of these proteins are appropriate, we can stimulate one of them and get emission from the other one: to have this phenomenon, the 2 molecules have to be very very close. When the scientist decide to excite the ciano protein and PKA is inactive, this cause the emission of fluorescent signal in the yellow range, not in the bleu.

When cAMP increases, PKA is activated, because the CFP do not interfere in the binding of cAMP with the regulatory domain. cAMP binds to the regulatory subunits and this causes the dissociation of the regulatory subunit from the catalytic one. The fluorescence proteins are far away, so we have no FRET: exciting the CFP, we'll have a blue emission.

Such engineering PKA demonstrate that cAMP increases because PKA is activated, so it is indirect; using Ca dyes we measure directly Ca.

In this way the scientist demonstrate also that the expression of cAMP is localized, not homogeneous. Red colors represent a region in which PKA is activated. Using this strategy, stimulate a cell sensitive for cAMP and discriminate the distribution of PKA. If the same experiment is done in the presence of inhibitor of PDE (fosfodiesterasi), enzymes which convert cAMP into AMP, the discrete organization of PKA disappear and PKAse appear activated in a homogeneous way, so PDE are required to confine the activation of PKA in discrete areas of the cell. If PDE are inhibit, we mis-regulate the cAMP signal. 

The hypothesis is that cAMP signals develop in a discrete way because all the machinery involved in the signal generation are confined and aggregate in discrete zones of the cell: this is the \emph{molecular channeling hypothesis}. 

Let consider a ligand and the receptor that is a G-protein, localized in discrete zones, and the AC that catalyses the production of cAMP: then an AKAP near to the mechanism that generate cAMP and this anchor the PKA in a discrete zone where the cAMP is produced. This PKA is the only one activated, because cAMP is generated in this macro-domain and AKAP anchor PKA and PDE: we have a concentration in this region of the cell of ON-mechanism and OFF-mechanism and the effector activated by cAMP. The substrate of PKA has to be very very close to this macro-domain.

How can cAMP reach the nucleus or other organelles, considering that these macro-domains are near the cell surface?

This hypothesis depends also on the concentration of cAMP in the macro-domain: if we prolong the stimulation of a specific ligand, like stimulate a cell per 10 seconds \lfreccia the receptor is activated and we reach a certain concentration of cAMP in the cell: this level is controlled by PDE. If the receptor is longer activated the concentration of cAMP in higher, so it reaches other AKAP proteins, that ca be localized ad the level of the nucleus membrane, causing the modulation of the gene expression.


It is also possible a generation of these macro-domains in internal zones of the cells: some g-coupled receptors activating AC are internalized in the target cells, so in the presence of some specific ligands, these receptor are internalized, and this is a normal answer to the ligand. They demonstrate this with antibodies for these receptors, so they proposed that if in the target cell an internal macro-domain of cAMP is required, this can also be produced by the internalization of the entire machinery. Sometimes a macro-domain that potentially is localized only at the level of the plasmalemma, after the internalization can be used to create macro-domains in the cytoplasm, to modulate other target molecules. To switch off this mechanism, maybe there will be a endosome fusing with the vesicle containing the machinery, but it is just a hypothesis.

All these evidences have been found out in cardiac cells, but also in neurons these domains can occur: the difficulty is that neurons are smaller and it is difficult to inject the engineered PKA, but this is difficult, so we should transform cells and induce them  to produce the engineered PKA.

\chapter{Cardiac muscle vs skeletal muscle}
\section{Cardiac muscle cells}
Heart has a low frequency, skeletal muscle has a high frequency, so the leves of Ca reached are different. The contractile activity of the heart is controlled and depends on situations: each single cell in the heart is characterized by a higher efficiency of contractile activity: the action potential in cardiac cells has to be transoformed in Ca signal, and this concvertion is not a all-or-none relation, but different Ca signals in terms of peak are reached depending on different action potential. 

Since Ca is a second messenger, in the ultrasctructural orgnization we have ON mechanism homogenously distributed in the cell, like the sarcoplasmi reticulum and the ryanodin-receptors type 2. The Ca signal will be synchronous in the cytoplasm and the contractile activity will be synchronous in all the cell.

Ca signals characterized by different amplitude: we can have amplifiers, because intracellular stores have receptors which can open by different concentration of Ca. At the level of the surface of the cardiac cells there are channels for Ca, Na and K: these voltage dependent Ca channels are localized also along the T-tubules membrane, so during the depolarization od the cardiac element, Ca enters into the cell as a consequence of the activation of the Ca voltage-dependent channels. This Ca reaches the SER, that have the ryanodin-receptors type 2: the signal will be amplified because this Ca influx in the cell is the signal for the opening of RyRs. For each Ca voltage-dependent channel, 4 RyR are open. 

During exercise, the Ca channel will prolong the duration of opening, so the Ca influx increase: more RyR will be activated, so more amplifiers will be recruited, and the level of Ca reach will be higher at rest.


In congenital heart failures (insufficienza cardiaca), the amplifiers are localized in wrong position: the cardiac cells can generate the action potential, a small Ca signal is generate, but not amplified, because the distance that Ca hat to cover for reach the RyR is to high.

\section{Skeletal muscle cells}
The main goal is to generate contractile activity in a synchronus way, but very very frequently: diffusion of Ca is not proper in this case!

In the triads, Ca channel directly interact with RyR on the SER: Ca events have to be generated very quickly! What is known is that for voltage dependent calcium channels, there are tetrads, composed by voltage dependent Ca channels, that control the RyR in the SER. In the T-tubule we are outside the cell, so one tetrad contacts a RyR, in the membrane of the SER, so 1 Ca channel for each subunit of the RyR 1(in cardiac is 1 for RyR, that is composed by 4 subunits). There is a intracellular loop in the voltage.dependent Ca channel that establish a bound with the cytoplasmic region of each subunit of the RyR1.


During the opening the Ca channels change configuration, that is enough to induce via the chemical bound, a rearrangement of each subunit of the RyR, that causes the opening of RyR1: so, RyR is opened by the opening of voltage-channel, not by Ca! they are called \emph{voltage sensors}.
In this case, the efflux of Ca from the SER is synchronous to the influx of Ca from the Ca-channels: that's why is it very very fast! We cannot control the amplitude of the Ca signal. 

RyR are 50\% coupled in tetrads, 50\% are not coupled: how these are activated? With Ca-induced-Ca-release? mmm... a new strategy proposes that the uncoupled RyR1 are actually in strict relation with the coupled one, so the uncoupled open synchronously with the coupled ones. There is a sort of cascade: when the tetrad is activated, the conformational change in coupled RyR1 causes a conformational change in the RyR1 uncoupled and their opening.

This strategy can be found also in some neurons, where we need the coupling between voltage-dependent Ca channels and RyR1 because we need a huge amount of Ca in some region and in a very very fast way. So, this mechanism is not a peculiarity of the skeletal muscle cells.






















\end{document}